\section{Conclusion}

This project was something which required in-depth view in django. Working on this project has taught me how to develop a social application. This project is  aimed at providing user a simple platform from which he/she can link his/her published book and popular social networking websites such as Facebook, Twitter and Instagram. For better user experience, there is change in colour of notification which will indicate any error. Working with the Open Source Community and a variety of people of different age group, one is always challenged by the fundamental difference between classroom coaching and real World experience. But such a challenge is exactly the purpose of six months training. The whole experience of working on this project and contributing to a few others has been very rewarding as it has given great opportunities to learn new things and get a firmer grasp on already known technologies. Here is a reiteration of some of the technologies I have encountered, browsed and learned:

\begin{enumerate}
    \item Operating System: Linux
    \item Language: Python, \LaTeX
    \item FrameWork: Django
    \item API: Faebook Graph API, Social Auth
    \item Softwares: Git, Doxygen
    \item Communication tools: IRC, Gitter
\end{enumerate}
So during this project, I learned all the above things. Above all, I got to know how software is
developed and how much work and attention to details is required in building even the most basic
of components of any project. Apart from above I also learned things like:
\begin{enumerate}
    \item Planning
    \item Designing
    \item Developing code
    \item Working in a team
    \item Testing
    \item Licensing Constraints
    \item How Open source community work?
    \item Writing Readable code
    \item Coding Standards
\end{enumerate}
And these are all very precious lessons in themselves.
