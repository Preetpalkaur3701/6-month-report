\section{Implementation}

\begin{enumerate}
\item Every workbench is nothing more than a folder containing an Init.py and/or InitGui.py. I just need to define FreeCAD commands, that can be made into menu items or toolbar buttons.

\item The inputs provided by user from rebar dialog box will be passed to our custom rebar function. This function will use the inputs to define the shape of rebar and in this function the sketcher object will be added to the FreeCAD active document which will hold the profile of the rebar by calculating coordinates of vertices and drawing the shape of rebar from user inputs.

\item Then this sketcher object and the selected structural object will pass to the prebuilt function of FreeCAD which will create the rebar. Below is the detailed description of that function.

\textit{makeRebar([baseobj,sketch,diameter,amount,offset,name]):} Adds a reinforcement bar object to the given structural object, using the given sketch as profile.

By following the above approach I can reuse the existing implementation of reinforcement system in the FreeCAD and at the same time proposing something new and interesting which will enrich user experience. Figure \ref{fig:flow}

\begin{figure}
    \centering \includegraphics[width=\linewidth]{images/RebaraddonDFD.png}
    \caption{Represenation diagram of data flow from Rebar addon to FreeCAD object}
    \label{fig:flow}
\end{figure}
\end{enumerate}

\section{Activation of Rebar Addon}
\begin{enumerate}
\item Open the FreeCAD Addon Manager (Tool \textrightarrow  Addon manager).
\item When an addon manager will open, select Reinforcement from a list of workbenches shown by an addon manager.
\item After selecting, click on Install/Update button.
\item Restart FreeCAD.
\item Now you will see different rebars in a drop-down list of rebar tools (Arch \textrightarrow Rebar tools \textrightarrow Different rebars).
\end{enumerate}
After activation, Rebar Tool is added in the FreeCAD Arch Workbench as shown in Figure \ref{mainFreeCADwindow}

\begin{figure}
    \centering \includegraphics[width=\linewidth]{images/r9b5l7K.jpg}
    \caption{FreeCAD Window}
    \label{mainFreeCADwindow}
\end{figure}

\section{Tools of Rebar Addon}
\subsection{Creates a Straight reinforcement bar in a selected structural element}

\subsubsection{Description}
The Straight Rebar tool allows user to create a straight reinforcing bar in the structural element as show in Figure \ref{straightrebar}.

\begin{figure}
    \centering \includegraphics[scale=0.35]{images/StraightRebar.png}
    \caption{Straight rebar reinforcement}
    \label{straightrebar}
\end{figure}

\subsubsection{How to use}
\begin{itemize}
\item Create a structure element
\item Select any face of the structure
\item Then select Arch Rebar Straight Rebar from the rebar tools
\item A task panel will pop-out on the left side of the screen as shown in Figure \ref{straightrebardialog}.
\begin{figure}
    \centering \includegraphics[scale=0.60]{images/StraightRebarDialog.png}
    \caption{Dialog box of Straight Rebar}
    \label{straightrebardialog}
\end{figure}
\item Select the desired orientation
\item Give the inputs like front cover, right side cover, left side cover, bottom cover and diameter of the rebar
\item Select the mode of distribution either amount or spacing
\item If spacing is selected, a user can also opt for custom spacing
\item Pick selected face is used to verify or change the face for rebar distribution
\item Click OK or Apply to generate the rebars
\item Click Cancel to exit the task panel
\end{itemize}


\subsection{Creates a UShape reinforcement bar in a selected structural element}
\subsubsection{Description}
The UShape Rebar tool allows user to create a UShape reinforcing bar in the structural element as show in Figure \ref{ushaperebar}.

\begin{figure}
    \centering \includegraphics[scale=0.35]{images/Footing_UShapeRebar.png}
    \caption{UShape rebar reinforcement}
    \label{ushaperebar}
\end{figure}
\subsubsection{How to use}
\begin{itemize}
\item Select UShape Rebar tool from the rebar tools.
\item A task panel will pop-out on the left side of the screen as shown in Figure \ref{ushaperebardialog}
\begin{figure}
    \centering \includegraphics[scale=0.60]{images/UShapeDialog.png}
    \caption{Dialog box of UShape Rebar}
    \label{ushaperebardialog}
\end{figure}
\item Click OK or Apply to generate the rebars.
\end{itemize}

\subsection{Creates a LShape reinforcement bar in a selected structural element}
\subsubsection{Description}
The LShape Rebar tool allows user to create LShape reinforcing bar in the structural element as shown in Figure \ref{lshaperebar}.
\begin{figure}
    \centering \includegraphics[scale=0.35]{images/LShapeRebarNew.png}
    \caption{LShape rebar reinforcement}
    \label{lshaperebar}
\end{figure}
\subsubsection{How to use}
\begin{itemize}
\item Select LShape Rebar tool from the rebar tools.
\item A task panel will pop-out on the left side of the screen as shown in Figure \ref{lshaperebardialog}
\begin{figure}
    \centering \includegraphics[scale=0.60]{images/LShapeDialog.png}
    \caption{Dialog box of LShape Rebar}
    \label{lshaperebardialog}
\end{figure}
\item Click OK or Apply to generate the rebars.
\end{itemize}

\subsection{Creates a Bent Shape reinforcement bar in a selected structural element}
\subsubsection{Description}
The Bent Shape Rebar tool allows user to create a bent shape reinforcing bar in the structural element as shown in Figure \ref{bentshaperebar}.
\begin{figure}
    \centering \includegraphics[scale=0.35]{images/BentShapeRebar.png}
    \caption{Bent Shape rebar reinforcement}
    \label{bentshaperebar}
\end{figure}
\subsubsection{How to use}
\begin{itemize}
\item Select Bent Shape Rebar tool from the rebar tools.
\item A task panel will pop-out on the left side of the screen as shown in Figure \ref{bentshaperebardialog}
\begin{figure}
    \centering \includegraphics[scale=0.60]{images/BentShapeDialog.png}
    \caption{Dialog box of Bent Shape Rebar}
    \label{bentshaperebardialog}
\end{figure}
\item Click OK or Apply to generate the rebars.
\end{itemize}

\subsection{Creates a Stirrup reinforcement bar in a selected structural element}
\subsubsection{Description}
The Stirrup Rebar tool allows user to create a stirrup reinforcing bar in the structural element as shown in Figure \ref{stirruprebar}.
\begin{figure}
    \centering \includegraphics[scale=0.35]{images/Stirrup.png}
    \caption{Stirrup rebar reinforcement}
    \label{stirruprebar}
\end{figure}
\subsubsection{How to use}
\begin{itemize}
\item Select Stirrup Rebar tool from the rebar tools.
\item A task panel will pop-out on the left side of the screen as shown in Figure \ref{stirruprebardialog}
\begin{figure}
    \centering \includegraphics[scale=0.60]{images/StirrupDialog.png}
    \caption{Dialog box of Stirrup Rebar}
    \label{stirrupshaperebardialog}
\end{figure}
\item Click OK or Apply to generate the rebars.
\end{itemize}

\subsection{Creates a Helical reinforcement bar in a selected structural element}

\subsubsection{Description}
The Helical Rebar tool allows user to create a helical reinforcing bar in the structural element as shown in Figure \ref{helicalshaperebar}.
\begin{figure}
    \centering \includegraphics[scale=0.35]{images/HelicalRebar.png}
    \caption{Helical rebar reinforcement}
    \label{helicalshaperebar}
\end{figure}
\subsubsection{How to use}
\begin{itemize}
\item Select Helical Rebar tool from the rebar tools.
\item A task panel will pop-out on the left side of the screen as shown in Figure \ref{helicalshaperebardialog}
\begin{figure}
    \centering \includegraphics[scale=0.60]{images/HelicalRebarDialog.png}
    \caption{Dialog box of Helical Rebar}
    \label{helicalshaperebardialog}
\end{figure}
\item Click OK or Apply to generate the rebars.
\end{itemize}

\subsection{Custom Spacing}
\subsubsection{Description}
The Custom Spacing tool allows a user to create rebar distribution in the structural element. You can define three segments for the distribution. For the first and third segments, you can give both a number of rebars and spacing between rebars. But for the second segment, you can only give either a number of rebars or spacing between rebars because one value automatically determines other.

For eg.: Given input values to Rebar Distribuiton dialog as shown in Figure \ref{rebardistributiondialog1}:

\begin{figure}
    \centering \includegraphics[scale=0.60]{images/RebarDistributionDialog.png}
    \caption{Dialog box of Rebar Distribution}
    \label{rebardistributiondialog1}
\end{figure}

Output produces by Rebar Distribution dialog when user click on OK button as shown in Figure \ref{rebardistribution1}:

\begin{figure}
    \centering \includegraphics[scale=0.35]{images/RebarDistribution.png}
    \caption{Rebar distribution of Stirrup reinforcement}
    \label{rebardistribution1}
\end{figure}

%

%Customizer  will provide User Interface to Customize Models interactively instead of modifying them manually. It will make the user able to create the templates for given model which can further be customized to cater to their need of different users and also provide a feature to save the set of parameters which define a different model using the same template of the model.
%\subsection{Activation of Customizer functions}
%\begin{itemize}
%   
%    \item This is experimental functionality.So, Initially
%    OpenSCAD will look like \ref{fig:Normal OpenSCAD}
%    \item In [Edit] menu, select [Preferencews] then open tab [Features], tick Customizer, then close the window when tick shown \ref{fig:3}.
%    \item In View menu, you shall now have an option [Hide customizer], that you shall untick. Then you will be able to see the customizer \ref{fig:OpenSCAD with Customizer}
%   
%\end{itemize}
%
%\begin{figure}
%       \centering \includegraphics[width=\linewidth]{images/output/2.png}
%       \caption{OpenSCAD without customizer}
%       \label{fig:Normal OpenSCAD}
%\end{figure}
%\begin{figure}
%       \centering \includegraphics[width=\linewidth]{images/output/3.png}
%       \caption{Preferences Widget to activate Customizer}
%       \label{fig:3}
%\end{figure}
%\begin{figure}
%       \centering \includegraphics[width=\linewidth]{images/output/5.png}
%       \caption{OpenSCAD with Customizer }
%       \label{fig:OpenSCAD with Customizer}
%\end{figure}
%
%\subsection{Syntax support for generation of the customization form}
%
%    \begin{lstlisting}[language=c++]
%    // variable description
%    variable name = defaultValue; // possible values
%    \end{lstlisting}
%
%Parameter will be decorated with meta data using the comments and the single line comments above the parameter could be used to describe the meaning of the parameter and its use. The comments in same line as that of parameter is used to define the GUI that have to be used to modify that values of that parameter.
%
%Following is the syntax for how to define different types of widgets in the form
%
%\begin{enumerate}
%    \item \textbf{Drop down box:} \ref{fig:example} Following type of comboBox could be created with following syntax:
%		\begin{figure}
%		\centering
%		\includegraphics[width=\linewidth]{images/example}
%		\caption{Shows the different types ComboBox, Slider}
%		\label{fig:example}
%		\end{figure}
%
%    \begin{lstlisting}[language=c++]
%    // combo box for nunber
%    Numbers=2; // [0, 1, 2, 3]
%   
%    // combo box for string
%    Strings="foo"; // [foo, bar, baz]
%   
%    //labeled combo box for numbers
%    Labeled_values=10; // [10:L, 20:M, 30:L]
%   
%    //labeled combo box for string
%    Labeled_value="S"; // [S:Small, M:Medium, L:Large]
%    \end{lstlisting}
%    \item \textbf{Slider:} Only numbers are allowed in this one, \ref{fig:example} specify any of the following:
%    \begin{lstlisting}[language=c++]
%    // slider widget for number
%    slider =34; // [10:100]
%   
%    //step slider for number
%    stepSlider=2; //[0:5:100]
%    \end{lstlisting}
%    \item \textbf{Checkbox:} \ref{fig:example1} Following widget is created with given syntax:
%    \begin{lstlisting}[language=c++]
%    //description
%    Variable = true;
%    \end{lstlisting}
%    \item \textbf{Spinbox:} \ref{fig:example1} Following widget is created with given syntax:
%    \begin{lstlisting}[language=c++]
%    // spinbox with step size 1
%    Spinbox= 5;
%    
%    // spinbox with step size 0.01
%    Spinbox= 5.11;
%    
%    //spinbox with given step size
%	SpinboxWithStep= 5; //2
%    \end{lstlisting}
%    \item \textbf{Textbox:} \ref{fig:example1} Following widget is created with given syntax:
%    \begin{lstlisting}[language=c++]
%    //Text box for vector with more than 4 elements
%    Vector=[12,34,44,43,23,23];
%   
%    // Text box for string
%    String="hello";
%   
%    \end{lstlisting}
%    \item \textbf{Special vector:} \ref{fig:example1} Following widget is created with given syntax:
%    \begin{lstlisting}[language=c++]
%   
%    //Text box for vector with less than or equal to 4 elements
%    Vector2=[12,34,45,23];
%    \end{lstlisting}
%	\begin{figure}
%		\centering
%		\includegraphics[width=\linewidth]{images/example1}
%		\caption{Shows the CheckBox, TextBox, SpinBox, VectorWidget}
%		\label{fig:example1}
%	\end{figure}
%
%\end{enumerate}
%
%\subsection{Creating Tabs}
%Parameters can be grouped into \textbf{tabs}. This feature will allow us to separate similar and related parameters. The syntax for this is also mainly similar to that of Thingiverse syntax for creating the tabs. To create a tab, use a multi-line block comment like this:
%
%\textbf{/* [Tab Name] */}
%
%
%Screenshot number \ref{fig:5} Shows the implemenation of this feature.
%
%The following tab names are reserved for special functionality:
%\begin{description}
%\item [Global] Parameters in the global tab will always be shown on every tab no matter which tab is selected. Note: there will be no tab for “Global” params, they will just always be shown in all the tabs.
%
%\item [Hidden] Parameters in the hidden tab will never be displayed. Not even the tab will be shown. Even though the variables who have not been parameterized using the Thingiverse or native syntax will not be displayed in OpenSCAD parameter widget but we have implemented this to make our comment like syntax similar as that of Thingiverse.
%\end{description}
%
%Also, the parameters who are under no tab will be displayed under TAB named “parameters”.
%\begin{lstlisting}[language=c++]
%    // combo box for nunber
%    Numbers=2; // [0, 1, 2, 3]
%   
%    // combo box for string
%    Strings="foo"; // [foo, bar, baz]
%   
%   
%    /*[ Slider ]*/
%    // slider widget for number
%    slider =34; // [10:100]
%   
%    //step slider for number
%    stepSlider=2; //[0:5:100]
%   
%    /* [Global] */
%   
%    //description
%    Variable = true;
%   
%    /*[Hidden] */
%   
%    // spinbox with step size 1
%    Spinbox = 5;
%   
%    /* [Textbox] */
%   
%    //Text box for vector with more than 4 elements
%    Vector=[12,34,44,43,23,23];
%   
%    // Text box for string
%    String="hello";
%   
%\end{lstlisting}
%
%\begin{figure}
%    \centering \includegraphics[width=\linewidth]{images/output/6.png}
%    \caption{Shows different groups generated through customzier}
%    \label{fig:5}
%\end{figure}
%
%\subsection{ Saving Parameters value in JSON file}
%This feature which is unique to openSCAD give the user the ability to save the values of all parameters and also we can apply them through the cmd-line and get the output.
%
%And JSON file is written in the following format:
%
%\begin{lstlisting}[language=Java]
%{
%    "parameterSets":
%    {
%        "set-name":
%        {
%            "parameter-name" :"value",
%            "parameter-name" :"value"
%        },
%        "set-name":{
%            "parameter-name" :"value",
%            "parameter-name" :"value"
%        },
%    },
%    "fileFormatVersion": "1"
%}
%\end{lstlisting}
%
%\textbf{Example:}
%\begin{lstlisting}[language=Java]
%{
%    "parameterSets":
%    {
%        "FirstSet":
%        {
%            "Labled_values": "13",
%            "Numbers": "18",
%            "Spinbox": "35",
%            "Vector": "[2,34,45,12,23,56]",
%            "slider": "2",
%            "stepSlider": "12",
%            "string": "he"
%        },
%        "SeconSet":
%        {
%            "Labled_values": "10",
%            "Numbers": "8",
%            "Spinbox": "5",
%            "Vector": "[12,34,45,12,23,56]",
%            "slider": "12",
%            "stepSlider": "2",
%            "string": "hello"
%        }
%    },
%    "fileFormatVersion": "1"
%}
%\end{lstlisting}
%
%You can write the JSON file using the two methods:
%\begin{itemize}
%    \item Manually writing the JSON file
%    \item Using the OpenSCAD's Customizer GUI
%\end{itemize}
%
%\subsection{Appling Parameters sets from JSON file}
%To select the parameter set from the JSON file and apply them on the model. We have two options:
%
%\begin{itemize}
%    \item Cmdline
%    \item GUI
%\end{itemize}
%
%\subsubsection{Cmdline}
%Cmdline option allows use for apply differenet set of parameters to the model without using the GUI and it also helps to use all the features provided by the customzier to be accessed using the cmdline. This feature will help other web-based softwares to utilize this feature. You can see various options available though cmdline in the figure \ref{fig:7}
% 
%\begin{lstlisting}[language=bash]
%openscad --enable=customizer -o model-2.stl -p parameters.json -P 
%model-2 model.scad
%\end{lstlisting}
%
%\begin{lstlisting}[language=bash]
%openscad --enable=customizer -o <output-file> -p <parameteric-file> -P 
%<NameOfSet> <input-file SCAD file >\end{lstlisting}
%
%\begin{itemize}
%    \item -p is used to give input JSON file in which parameters are saved.
%    \item -P is used to give the name of the set of the parameters written in JSON file.
%\end{itemize}
%
%\begin{figure}
%    \centering \includegraphics[width=\linewidth]{images/output/8.png}
%    \caption{Cmdline options available for customizer}
%    \label{fig:7}
%\end{figure}
%
%\subsubsection{GUI}
%
%Through GUI you can easily apply and save Parameter in JSON file using Present section in Customizer explained below.
%
%In customizer, You will be able to see two checkbox’s which are
%\begin{itemize}
%\item \textbf{Automatic Preview:}
%If checked preview of the model will be automatically updated when you change any parameter in Customizer else you need to click preview button after you update parameter in the customizer.
%\item \textbf{Show Details:}
%If checked the description for the parameter will be shown above the input widget for the parameter else It will not be displayed but you still can view the description by hovering the cursor over the input widget.
%\end{itemize}
%Then comes \textbf{Reset} button which when clicked resets the values of all input widgets for the parameter to default provided in SCAD file.
%
%Next, come Preset section: It consist of three buttons
%\begin{description}
%    \item \textbf{Combo Box:}
%        It is used to select the set of parameters to be used
%    \item \textbf{+ button:}
%    \begin{enumerate}
%        \item It is used to update the set selected in combo Box. On clicking + button values of parameters in set are replaced by new values
%        \item If we select “No set selected” in comboBox, then we can use + button to add new set of the parameters \ref{fig:Add new set}
%    \end{enumerate}
%    \begin{figure}
%        \centering \includegraphics[width=\linewidth]{images/output/7.png}
%        \caption{Widget to add new set to JSON file}
%        \label{fig:Add new set}
%    \end{figure}
%
%    \item \textbf{$-$ button: }
%        It is used to delete the set selected in combo Box.
%    and finally below Preset Section is the Place where you can play with the parameters.
%\end{description}
%
%
%You can also refer to  two examples that are Part of OpenSCAD to learn more
%\begin{enumerate}
%    \item Parametric/sign.scad
%    \item Parametric/candlStand.scad
%\end{enumerate}
%
%After running a set of commands of \LaTeX and SageMath, it produces the
%output in PDF form (with pdflatex) Figure \ref{fig:5}.
%
%\begin{figure}
%\centering \includegraphics[width=\linewidth]{images/output/9.png}
%\caption{customizer with different set of parameters selected}
%\label{fig:8}
%\end{figure}
