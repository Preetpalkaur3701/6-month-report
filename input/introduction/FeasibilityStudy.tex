\section{Feasibility Study}
Feasibility study aims to uncover the strengths and weaknesses of
a project. In its simplest term, the two criteria to judge feasibility
are cost required and value to be attained. As such, a well-designed
feasibility analysis should provide a historical background of the
project, description of the project or service, details of the
operations and management and legal requirements. Generally, feasibility
analysis precedes technical development and project implementation.
These are some feasibility factors by which we can determine that
the project is feasible or not:
\begin{itemize}
\item {\bf{Technical feasibility}}: Technological feasibility is carried
out to determine whether the project has the capability, in terms of
software, hardware, personnel to handle and fulfill the user requirements. This whole project is based on Open
Source Environment and is part of an open source software which would be deployed on any OS.

\item {\bf{Economic feasibility}}: In Economic feasibility, we
determine whether the benefit is gain according to the cost invested
to develop the project or not. If benefits outweigh costs, only then
the decision is made to design and implement the system. It is
important to identify cost and benefit factors, which can be categorized
as follows:

\begin{enumerate}
\item Development costs.
\item Operating costs.
\end{enumerate}
Thepoet.me is also Economically feasible for a year as It could be developed and maintain with zero cost as It is supported by Open source community and GitHub Students Pack.
\end{itemize}


