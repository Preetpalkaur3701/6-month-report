\section{Objective of Project}

To ease up the rebaring process in FreeCAD, an interactive addon is developed where user will input the required data as per the design requirements and they will need not to draw rebars from the Sketcher workbench for creating reinforcement in the structural object.

One of the primary benefits of this addon is the ability to create parametric rebars. These are designs which are parametrized using parameters or top-level variables.

\begin{enumerate}
    \item Provides full GUI (Graphical User Interface) for reinforcing rebars in the structural object.
    \item They also requires pre-built standard rebar shapes.
    \item Different rebars will have their own view and data properties.
    \item User can edit the parameters of the group of rebars from the view \& data properties itself.
    \item These rebars will be fully parametric. Hence the parameters of the rebars will automatically adjust themselves if the changes are made to their parent structure.
    \item Easy to use. 
\end{enumerate}


%The major Objectives of this project are:
%\begin{enumerate}
%    \item Syntax support for generation of customization form:
%    The customization form generated on Thingiverse is based on a certain syntax for both describing the elements in the form and providing a range of their values. In order to make this work in OpenSCAD as well, the same style of description and parameterization can be incorporated into OpenSCAD. Hence the user will be able to generate the customization form from within OpenSCAD by adding a few simple lines in the .scad file.
%  
%    Customization of the model from the form:
%    Once the form is ready, it must be able to customize the model as desired by the user. The changes made in the form should directly correspond to changes in the model itself.
%    Enhancing the UI for the customization form:
%    The customization form is there to make the whole customization thing easy. And that implies that the form itself should also be easy to use. And this can be achieved by having a good and simple look to the whole thing.
%  
%\end{enumerate}

