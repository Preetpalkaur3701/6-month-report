\section{User Requirement Analysis}
User Requirements Analysis for a software system is a complete description of the requirements of the User. It includes functional Requirements
and Non-functional Requirements. Non-functional requirements are
requirements which impose constraints on the design or implementation.

 
{\bf Users of the System:}
    \begin{enumerate}
        \item Provides full GUI (Graphical User Interface) for reinforcing rebars in the structural object.
        \item They also requires pre-built standard rebar shapes.
        \item Different rebars will have their own view and data properties.
        \item User can edit the parameters of the group of rebars from the view \& data properties itself.
        \item These rebars will be fully parametric. Hence the parameters of the rebars will automatically adjust themselves if the changes are made to their parent structure.
        \item Easy to use. 
               
    \end{enumerate}

\subsection{Functional Requirements}
\begin{itemize}
    \item  {\bf Specific Requirements}: This phase covers the whole requirements for the system. After
    understanding the system we need the input data to the system then we watch the output
    and determine whether the output from the system is according to our requirements or not.
    So what we have to input and then what well get as output is given in this phase. This phase
    also describe the software and non-function requirements of the system.

    \item {\bf Input Requirements of the System}:
    \begin{itemize}
        \item{\bf Straight Rebar}: Orientation, Front Cover, Right Cover, Left Cover, Cover along, Bottom Cover, Top Cover, Amount and Spacing.
        \item{\bf UShape Rebar}: Orientation, Front Cover, Right Cover, Left Cover, Bottom Cover, Top Cover, Rounding, Amount and Spacing.
        \item{\bf LShape Rebar}: Orientation, Front Cover, Right Cover, Left Cover, Bottom Cover, Top Cover, Rounding, Amount and Spacing.
        \item{\bf BentShape Rebar}: Orientation, Front Cover, Right Cover, Left Cover, Bottom Cover, Top Cover, Anchor Length, Bent Angle, Rounding, Amount and Spacing.
        \item{\bf Stirrup}: Front Cover, Right Cover, Left Cover, Bottom Cover, Top Cover, Bent Angle, Bent Factor, Rounding, Amount and Spacing.
        \item{\bf Helical Rebar}: Side Cover, Top Cover, Bottom Cover, Pitch and Diameter.
        \item{\bf Rebar Distribution}: Number of rebars and Spacing in segment 1, segment 2 and segment 3.
     \end{itemize}

    \item{\bf Output Requirements of the System}: Reinforced structure is produced which contains user selected rebar shapes. 

    \item {\bf Interactive mode}: The user can operate rebar addon from an interactive mode of FreeCAD.

    \item \textbf{User interaction mirroring on the console}: Everything the user does in the FreeCAD interface executes Python code, which can be printed on the console and recorded in macros.

\end{itemize}
\subsection{Non-functional requirements}
\begin{enumerate}
    \item Extensible: It should be able to support future functional requirements
    \item Usability: Simple user interfaces that a layman can understand.
    \item Modular Structure: The software should have  structure. So, that different parts of software would be changed without affecting other parts.
\end{enumerate}

