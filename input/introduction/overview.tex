\section{Overview}

Thepoet.me(Usually styled as thepoet.me)  is a Django based site that I worked upon for my 6-month training. The site offers registered user a simple platform from which to link his/her published book and popular social networking websites such as Facebook, Twitter and Instagram. It is characterized by its one-page user profiles, each with a large, often-artistic background and abbreviated biography. It is a free and Open-source application. The site is fully responsive with all type of devices.

Your thepoet.me acts as a virtual online business card. Put the URL to your site in your e-mail as digital signature, Twitter profile, share it on Facebook, add it to your Instagram as a website.

If you are a poet or writer of some sort who doesn't have a website, you can point colleagues, clients, and prospects to your thepoet.me page so they can find out more about you and connect with you in all the right places.

There are countless platforms out there that you can use to build your own free personal website, but not all of them will deliver the same sense of quality and professionalism. If you’re a poet or writer and looking for something fast and simple that you just need to represent a landing page for yourself, thepoet.me could be one of your best alternatives to choose from.

Other website and blog building tools like Blogger and WordPress.com offer a complete platform to build upon, including the ability to host several web pages, write blog posts and feature widgets. thepoet.me gives you just one, single page to display all your links and a summary of yourself, making it an ideal tool to get straight to the point about you and your published books.

The main idea of this project is to provide a personal identity page to poets or writers with features to link their published book and social networking sites. User Interface is designed in a way such that layman can easily understand it. The core part of this project is implemented using Django and for GUI part Bootstrap is used. My training being not based on particular language or technology, different types of open-source software’s and technologies are used in this project and many during my training which are not used in this project like Android for Nitnem, Jekyll for blog and shell scripting for plzalert.me

My training being not based on particular language or technology, different type of open-source software's and technologies are
used in this project and many during my training which are not used in this
project like Django, Facebook's Graph API for\emph{plzalert.me} WebApp.
